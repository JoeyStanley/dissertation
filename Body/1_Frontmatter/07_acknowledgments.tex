\setcounter{page}{4}

\chapter*{Acknowledgments}
% chapter* removes chapter number, but also removes from TOC, so we add it back
\addcontentsline{toc}{chapter}{Acknowledgments}

While my name is associated with this dissertation, this work was not done single-handedly. Numerous people have contributed their skills, knowledge, and expertise to help me complete this project. Regardless of the size of their contribution and whether they're even aware of their help, it is worth mentioning just some of those who have helped me along the way.

Many of my undergraduate professors at Brigham Young University played an important role in helping me get to where I am today. Because of where my last name falls in the alphabet, Alan Melby was my default advisor and when I met with him soon after becoming a linguistics major, he (easily) convinced me to minor in linguistics computing. He and Tyler Snow later introduced me to Perl, but it was Monte Shelley who took me under his wing and provided an environment where I my Perl skills could really grow. While these computational skills developed, Janis Nuckolls was the one that mentored me and began to shape me into a bona fide linguist. She organized the study abroad to Ecuador and invited me to be a part of her research team. Ultimately, my research interests shifted away from indigenous languages of South America, but my constant drive for submitting papers to conferences certainly came from working with her for two years.

Several of my friends and peers have also helped me along the way, including Mike Olsen, Rachel Olsen, Rachel Kim, Andrew Bray, Lisa Lipani, Kyle Vanderniet, and Katie Kuiper. Our discussions provoked questions that ultimately led to some of the methods and results in this project. Jonathan Crum is 100\% responsible for converting me over to \LaTeX{} and the Tufte-inspired layout I use here wouldn't have been possible had I stuck with Microsoft Word. I'm grateful also to Rich Ross for being my personal on-call statistician. I hope my random texts at all hours of the day (``Is \textit{this} right?'' or ``It's a bad idea for me to do \textit{that}, right?'') weren't too annoying. Sorry for talking shop so much in front of your family.

Most directly related to this project, I humbly acknowledge the support of two specific funding sources that helped me along the way. First, the University of Georgia Graduate School Dean's Award provided the funds to travel to Washington and conduct fieldwork. And second, the University of Georgia Graduate School Summer Doctoral Research Fellowship allowed me two months of uninterrupted time during which I completed the bulk of my transcriptions. I will likely never know who the donors are that that money came from, but their contributions turned my project into a reality. (Also, I'm grateful to Chad Howe for informing me of these funding opportunities and helping me with my applications.)

To put it quite frankly though, this project would not have been possible without the help of my mother-in-law, Cathy Jones. She (and my father-in-law) opened their home to me, my wife, and our two-week old daughter while I conducted interviews for over a month. Not only that, but she called over 100 people searching for natives to the area, served as my secretary (taking phone calls, arranging my calendar, driving me to the furthest corners of Cowlitz County), and participated in a large number of the interviews herself. Ultimately, I interviewed 34 people through her recruitment efforts. Without her, I would have returned home with maybe two dozen (very stilted) interviews and called the project a failure.

I thank my committee members for their guidance during graduate school. Chad Howe was my guide and mentor as I began my journey into the field of sociolinguistics, particularly when navigating my first NWAV. A lot of the interpretation of my findings here came from long chats in his office. Peggy Renwick has been with me every step of the way as I developed my statistics and coding skills, beginning with first introducing me to R in her Quantitative Methods course. After a few years she invited me to join her in learning to tackle generalized additive models, which became the primary analytical tool in this project. Bill Kretzschmar provided employment at the Linguistic Atlas Project for four years where I could develop new skills (like Shiny) and participate in collaborative research, culminating in numerous presentations and papers. I was glad to be able to pick his brain walking the streets of Gouda, eating pizza along the \textit{gracht} in Utrecht, and getting honked at while (probably illegally) walking in the grass along some Dutch highway. These three served as stakes to a young tree: their knowledge and experience guided me as I developed but still allowing me to grow in my own direction.


Then there are people that I have never met, but who have freely provided materials online for others to learn.
\begin{itemize}
    \item For Praat scripting, I thank J\"org Mayer and their website \textit{Phonetics on Speed: Praat Scripting Tutorial}\footnote{http://praat\-scripting.ling\-phon.net}, Will Styler and his \textit{Save the Vowels} tutorial\footnote{http://save\-the\-vowels.org/praat/}, and the various individuals, univerisities, and linguistics labs who have Praat scripts available. I'm particularly grateful for those written by the members of the University of Washington Phonetics Lab and Mietta Lennes for the Speech Corpus Toolkit for Praat \citep[SpeCT; ][]{lennes_2017}. I selfishly used these scripts to Frankenstein my own together.
    \item For general R coding, I again thank Peggy Renwick for sending me example code and tolerating my incessant questions. Josef Fruehwald indirectly (and unknowingly) introduced me to the Tidyverse, Paul Tol's color schemes, and visualizations of GAMMs. I'm hugely grateful for Garrett Grolemund and Hadley Wickham for writing the book \textit{R for Data Science}\footnote{https://r4ds.had.co.nz} and making it such an approachable and free resource (and completing my conversion to the tidyverse). Finally, I thank Yihui Xie for coming up with R Markdown: what an incredible organizational tool for R code!
    \item I make extensive use of GAMMs, and none of that would be possible without several people. Peggy Renwick initially introduced them to me (I think around April 2017) as we worked through Mart\'on S\'oskuthy's \citeyearpar{soskuthy_2017} especially thorough tutorial. Jocolien van Rij, Bodo Winter, and Josef Fruehwald's excellent tutorials and example code were also most helpful. Peggy then invited me to collaborate on a paper \citep{renwick_stanley_2020} that would use GAMMs at \textit{just} the right time, and we were able to learn from Herold Baayen, who volunteered much of his time to pass his knowledge to us.
\end{itemize}
I am indebted to each one of these people for producing such quality materials. It is a hallmark to the R community and the field of data science generally that so much can be learned through free, web-based materials. I hope the tutorials I have produced on my website in the meantime will one day begin to make as much a contribution as these people's work has.

I thank Kelly, Lena, and Walter for their support over the past several years. Kelly has been my on-call, resident Cowlitz County native, helping me confirm my findings regarding the culture of the area. She has also put up with years of me chugging away on my laptop at home, often late into the night. Not to mention incessent conversation about the linguistics of her hometown. Lena was just two weeks old when we flew out to Cowlitz County to conduct fieldwork---thanks for not crying too often in the middle of the night. As I submit this thesis, Walter is only six weeks old, and I'm appreciative of his health and easy-going disposition which allows me to concentrate as I put the final revisions on this document. What a blessing it is to have a supportive family while in graduate school!

Finally, I thank the residents of Cowlitz County who welcomed me into their lives to help me with my project, including the many others along the way who helped me find contacts. I've learned through this project that \textit{everyone} has a story to tell and I am grateful to have heard so many of them from such wonderful people. 
