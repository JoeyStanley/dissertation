\setcounter{page}{0}


\vspace*{18pt}
\begin{center}
  \textsc{\Large{\dissertationtitle}}\\[18pt]  % A little larger than the original template
  by\\[18pt]
  \textsc{\Large\myname}\\[12pt]
  (Under the Direction of L. Chad Howe)\\[12pt]
  \textsc{Abstract}
\end{center}


In this study, I present acoustic data on vowel formant dynamics from 54 natives of Cowlitz County, Washington, a region where English has not been analyzed sociolinguistically. The Elsewhere Shift, defined here as the approximation of the low back vowels and the retraction and/or lowering of the front lax vowels, is widespread across North America. The purpose of this study is to not only document the Elsewhere Shift in southwest Washington, but to also highlight change and variation in the formant curves of its vowels in addition to their position in the F1-F2 space.

Using data gathered through sociolinguistic interviews, I use generalized additive mixed-effects models to analyze change formant trajectories between generations and sexes. In the elsewhere allophones of \trap, \dress, and \kit, the onsets of the vowels retracted and/or lowered in the vowel space, resulting in a change from a "U-shape" trajectory to a "bounce"-shaped curve. Among the prenasal allophones, only \ban shows meaningful sociolinguistic change, suddenly raising and diphthongizing in Millennial women's speech. \lot and \thought have been in a stable near-merged state for four generations.

This study supports the proposed structural relationship between the five vowels and the pull chain hypothesis. The approximation of the low vowels was complete by the 1930s, \bat began shifting by the 1930s, \bet followed suit with the Boomers, and then the Millennials began retracting \bit in the 1980s. The timing of the changes correlate with cultural and demographic shifts in the community, namely the settlement of the area, the establishment of Long-Bell Lumber Company, the economic recession in the 1980s, and shifting attitudes about the community.

This study not only fills a small gap in our current dialectological map of North American English, but it also confirms that speaker orientation towards or away from a particular place manifests itself in speech patterns. Furthermore, it provides the first extensive look at the formant trajectories of the vowels involved in the Elsewhere Shift, illuminating sources of variation that would not be easily identified in a single-point analysis.


\thispagestyle{empty}

\begin{list}{\textsc{Index words:\hfill}}{\labelwidth 1.2in\leftmargin 1.4in\labelsep 0.2in}
  \item \begin{flushleft}
    sociolinguistics, vowel shifts, vowel formant dynamics, dialectology, Pacific Northwest English, generalized additive mixed-effects models
  \end{flushleft}
\end{list}
