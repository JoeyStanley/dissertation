
\section{Attitudes towards English in Cowlitz County}

In the sociolinguistic interviews, the topic of language often came up, and many people had some strong feelings about how people talk in Cowlitz County. However, they can be categorized into two broad groups



% Really small things that everyone does

Many of the comments related to speech patterns in Cowlitz County were relatively neutral and were about speech patterns that are certainly not unique to the area. For example, one speaker said that people use a lot of slang, though she was unclear how slang was defined. Donna \ref{quote:enunciate} and \ref{quote:yeah} for example, finds herself speaking quickly and feels that \textit{yeah} is somehow less correct thatn \textit{yes}:
\begin{num_quote}
    Sometimes I guess I will, um. I speak pretty quickly and I think that I run things together a lot. I don't enunciate. (Donna, F. b. 1968)
    \label{quote:enunciate}
\end{num_quote}
\begin{num_quote}
    Um so yeah I. \textit{Yeah}, That's one of my biggest things. I could say \textit{yes}, but I like saying \textit{yeah}. (Donna, F. b. 1968)
    \label{quote:yeah}
\end{num_quote}
Speaking quickly, using \textit{yeah}, and slang are not unique to Cowlitz County by any means. But these comments suggest that people do think about and notice differences in language.

In some cases, I was surprised at the kinds of speech patterns people picked up on. For example, two different speakers \ref{quote:rich_catch} and \ref{quote:donna_catch} made specific mention of the word \textit{catch}:
\begin{num_quote} % normal thing, neutral
    [As a part of the reading passage] ``\ldots This way, bragged the mice, the mean old cat would never c[\textipa{\ae}]tch them.'' C[\textipa{\ae}]tch them. Not c[\textipa{E}]tch them. (Rich, M, b. 1957)
    \label{quote:rich_catch}
\end{num_quote}
\begin{num_quote} % normal thing, negative
    When I was in high school, a friend moved here from New York. And he made fun of the way I said- I used to say \textit{c}[\textipa{E}]\textit{tch}. And so ever since then I've always tried to make myself say \textit{c}[\textipa{\ae}]\textit{tch} cuz he said I was saying it wrong. 
    
    \textit{I still say} c[\textipa{E}]tch.
    
    \textit{C}[\textipa{E}]\textit{tch}. Yeah. I mean, yeah, and I sometimes I will. That's one of those slippages. I'll try to say \textit{c}[\textipa{\ae}]\textit{tch} but sometimes it comes out as \textit{c}[\textipa{E}]\textit{tch}. I know there's no E in it, he goes, ``There's no E in it!'' (Donna, F. b. 1968).
    \label{quote:donna_catch}
\end{num_quote}
I believe this is telling of the attitudes towards English in Cowlitz County. In other varieties of English, people have assigned stigma to a variety of speech patterns, including phonological, morphosyntactic, and lexical variants. Of the countless linguistic variables that Cowlitz County speakers could have stigmatized, the realization of the vowel in \textit{catch} is the one that stands out to these people. If this is the thing that they notice the most in people's speech, other speech patterns must not have very many negative associations. Or these people simply do not notice variation in their community. \citet{evans_2013} has shown that many Washingtonians do not think there are linguistic differences in their state. To these Cowlitz County residents, there is an ideology of linguistic homogeneity in their community, and despite the variation that does exist (see the next three chapters), they only notice the vowel in the word \textit{catch}.

Rich made additional commentary about language, and demonstrates that he has a clear sense of what is a ``proper'' pronunciation. In \ref{quote:applicable}, he bemoans the changing acceptance of the word \textit{applicable}, but acknowledges that language changes and that he should accept the incoming variant.
\begin{num_quote} % normal thing, prescriptive, neutral
    And then there's the word- and one of my favorites: \textit{\'{a}pplicable}\footnote{Accent marks indicate the stressed syllable in this passage.}. If you think something is \textit{\'{a}pplicable}- because our living language is- everybody says \textit{appl\'{i}cable}. And it \textit{used} to be, if you looked in the dictionary the only way to pronounce it was \textit{\'{a}pplicable}. That is \textit{\'{a}pplicable}. So I've watched that one go away because \textit{appl\'{i}cable} shows up in the dictionary now. And I have to accept it, right? Because we've said that's going to be considered a proper pronunciation. (Rich, M, b. 1957)
    \label{quote:applicable}
\end{num_quote}
Rich also points out variation in other words, citing the dictionary as support for nonmainstream realizations:
\begin{num_quote} % normal thing, prescrptive, neutral
    [As a part of the reading passage] ``\ldots but the cat simply hid in the corner behind a sack of b[\textipa{e\textsubarch{I}}]gels.'' Oho, \textit{b}[\textipa{e\textsubarch{I}}]\textit{gels}! You're trying to see how I pronounce \textit{b}[\textipa{e\textsubarch{I}}]\textit{gels}.

    [After the reading passage was over] \textit{You're the first one to catch on!}
    
    You mean saying \textit{b}[\textipa{e\textsubarch{I}}]\textit{gels}? Do you know how to say \textit{b}[\textipa{e\textsubarch{I}}]\textit{gels} properly? I'm think I'm saying it the way people would consider it proper.
    
    \textit{I would call them} \textit{b}[\textipa{e\textsubarch{I}}]\textit{gels}.
    
    \textit{B}[\textipa{e\textsubarch{I}}]\textit{gels}. Mm-hm. B-A-Y-G-U-L-L-S is the only way I learned to say it properly when my wife called me on it. Cuz I used to say \textit{b}[\textipa{\ae}]\textit{gels}. Maybe it was something like that, I probably shouldn't try to figure it out cuz I'll go back to saying it. But it's that and C-R-E-E-K is \textit{cr}[\textipa{I}]\textit{k}. But if you look in the dictionary \textit{cr}[\textipa{I}]\textit{k} is there as an alternative pronunciation so you cannot correct me if I'm in the dictionary. (Rich, M, b. 1957)
    \label{quote:bagels}
\end{num_quote}
The words \textit{applicable}, \textit{bagel}, and \textit{creek} are not particularly common\footnote{Their frequency in COCA was 6.96, 3.19, and 39.85 tokens per million words, respectively \citep{davies_2008_coca}.}. If these are the biggest pet peeves that Rich has about language, he is not often bothered by them.

Other people have some more negative attitudes towards language in their community. Carla in particular, who works in education, had particularly strong feelings about some linguistic variants:
\begin{num_quote} % normal thing, negative
    \textit{Have you noticed anything about the language around here?}
    
    Um, that I want to strangle everyone that says the word \textit{seen}. \textit{Seen}. ``I seen it yesterday.'' [retching noise in disgust]. Um, in fact my friend who's a speech pathologist, that was our big thing in college we used to laugh about. ``I seen  ([\textipa{A: sE\textsubarch{I}n}]) that yesterday.'' It's a horrible, horrible grammar \textit{faux pas} here. \textit{Seen}. Oh I hate it. Yeah. \textit{Seen} is definitely the biggest one. (Carla, F, b. 1975)
\end{num_quote}
\begin{num_quote} % normal thing, negative
    I know um, a lot of people truncate their words here. They won't finish the word. They'll let it um, like, ``I was driving'' ([\textipa{a: z dZ\*ra\textsubarch{I}vIn}]). Or, ``eating'' ([\textipa{iP\s{n}}]). That bugs me. (Carla, F, b. 1975)
\end{num_quote}
These are not linguistic patterns unique to the area.



    

Teresa has what appear to be very negative perceptions of the English spoken in Cowlitz County. 
\begin{num_quote} % regional things, negative, education, 
    First of all, people who don't read much have a better local dialect. Yeah it- people who don't travel much who don't get out of the area much have a- have a-. Cuz I find that my language is different than a lot of people up-. If I'm in the grocery store sometimes I'm like, [whispered] what the--?

    \textit{What kinds of things specifically have you noticed about their speech?}
    
    Poor grammar for one thing. Y'know, that bugs crud out of me. So, ``I ain't got no.'' Um. You can tell if people have come from other places by the way they certain words, like \textit{out} or \textit{milk} or um. It's [\textipa{mElk}] for some people. It's \textit{Wa}[\textipa{\*r}]\textit{shington}. And it's like you don't know where they pick it-. I- I don't know where they're- where- where they're from particularly but. We do have a lot of people now that are from the South and we have um. I don't know, \textit{cr}[\textipa{I}]\textit{k} rather thank \textit{cr}[\textipa{i:}]\textit{k}. I grew up saying \textit{cr}[\textipa{I}]\textit{k} but I no longer say that. So that's something that had changed for me. So if I found a better way to say something then I would change my speech. (Teressa, F, b. 1956)
\end{num_quote}





\begin{num_quote} % normal thing, neutral, education
    Y'know, it's funny when you say that because I'm from here, I have a degree in English so I learned how to speak properly, I tend to try to speak properly, but I will slip into, y'know depending on who I'm with, I'm still from here. Y'know?
    
    \textit{What kinds of things do you notice yourself slipping into?}
    
    Oh gosh, that's a good question. Just- I don't know I mean I- you might hear me say an \textit{ain't} but it's not, I mean it's not cuz I don't know any better. I almost do it sarcastically. 
    
    \textit{Is there any sort of, y'know, speech things around town that you hear and you just cringe?}
    
    It- yes, it- it- and it's just because I-. To me it just sounds ignorant y'know. But, at the same time I understand that not everybody has the opportunity y'know to become better educated. 
\end{num_quote}




% Why do people sound different?

\begin{num_quote}
    So that's, y'know, or I think television too has changed our speech and and um. I think people have kind of adapted a uh and maybe I didn't notice it when I was a kid but but like the the redneck hillbilly style of speaking I don't remember that but sort was like well that's our identity so we're gonna talk like that yeah we're gonna identify we're gonna create a tribe and this is how we talk. (Teressa, F, b. 1956)
\end{num_quote}


\begin{num_quote}
    You may not find it but um, I notice that- that people who are conservative tend to be more colloquial. People who are more liberal tend to be more- but then again that's a part of it is cuz they left, they went off to college somewhere, got ``liberalized,'' and they came back, y'know. And- and I don't want to say Democrat Republican cuz it's not quite like that\ldots Um but yeah I see that I see that with people.  (Donna, F. b. 1968)
\end{num_quote}




% Is there an accent?

\begin{num_quote}
    \textit{Now that you've lived outside of Washington, has anybody said anything about the way that you speak?}
    
    Yes. So they say I talk different but they don't really have a word for it. Cuz like you can- you know like the New York accent or the Boston accent or southern but it's like ``what's Washington?'' But they know it's different but it's just not like termed very well I guess. (Kayla, F. b. 1993)
 \end{num_quote}
 
 
 
\begin{num_quote}
    Yeah I've heard people say things that are different. I can't come up with any specifics, um but, yeah I mean the whole \textit{potayto} \textit{potahto}. Some people are- they they just talk a little differently. I don't know if I notice it so much here unless they're from somewhere else. (Holly F, b. 1976).
\end{num_quote}


\begin{num_quote}
    \textit{Is there anything that you've noticed about the speech around here? Do you think there's much of a dialect or an accent?}
    
    Hard to say because I've lived here my whole life and I haven't really traveled much um so for me, y'know, like you said before we began, I'm more of a, ``hey you sound like you're from Boston'', ``hey you sound like you're from New York'' it- y'know? Whereas I've never heard anybody say, ``You sound like you're from Seattle,'' y'know. So, I mean we must, I mean, we- we must have a different dialect. But how it would compare to others?\ldots (Shane, M. b. 1971)
\end{num_quote}









\section{Attitudes towards English in Cowlitz County}

It might be good to compile and discuss what people think about English around here. If people have high linguistic security, this would be the section to mention that in. Hall-Lew \citep[75]{hall_lew_2009_diss} says that when asked about language, people usually point to prosodic things. 

\begin{quote}
The logger's speech is richer than the language of other occupations, not because of his invention of esoteric terms, but because of his manipulation of the English language and his flair for trenchant phrases. \citep[114]{davis_1950}
\end{quote}

\begin{quote}
Loggers are as well aware of the distinctive qualities of the speech as any student of language, and they enjoy coining new figures of speech… This interest in their own speech has kept it from becoming a reiteration of stereotypes, and has imparted to it an elusive, indefinable quality, a humorous twist, making it reflect the logger's personality in a way that is difficult to capture and set down on paper. \citep[114]{davis_1950}
\end{quote}




\bat had been lowering for at least two generations before \ban started lowering. But, as \bat continues to lower, \ban then rose.







Daniel says that jojos are potato wedges. Foreigners.

Andrew says hicks talk slower.

Doug talks about Warshington but didn't notice anything else. Foreigners.

Marilyn talks about Indian accent.

Holly also talks about how people should use less slang, more proper words.






%When asked about language patterns specifically, most people had very little to say. One exception was Adam \ref{quote:oh_my_gawd}, who described speech patterns in California.
%\begin{num_quote}
%    There’s- there’s this accent in California that I can’t quite put my finger on. And I don’t know if it’s just a Millennial accent or if it’s like, uh, Californian. But, um, there’s like this- this this female accent where it’s like “Oh my god!” ([\textipa{o: mAI gO{\super @}d}]\footnote{As far as I can tell, this impression lacked any vowel-related California features.}). Like they talk like that. It’s very nasally. I hear that a lot and I- I see that- I hear it all over the place. It seems like it’s not just West Coast or whatever but.
    
%    \textit{Do you hear that as far north as Portland?}
    
%    Ooh\ldots I haven't heard it in Portland. (Adam, 32, M)
%    \label{quote:oh_my_gawd}
%\end{num_quote}
%As described in \S\ref{sec:social_meaning_elsewhere_shift}, non-linguists have been aware of speech patterns in California English for several decades \citep{hinton_etal_1987}, and even within the state of California people pick up on regional patterns, \citep{bucholtz_etal_2007, bucholtz_etal_2008, villarreal_2018}, sometimes labelling speakers who such such varieties ``surfers'' or ``valley girls'' \citep{villarreal_2016_pads}. Adam has negative feelings about these speech patterns, but, despite his frequent visits to Portland, he has not noticed them there, even though they are to some extent \citep{becker_etal_2016_pads}. It appears then that the social meaning associated with variants of the Elsewhere Shift, as far as what speakers index and how those variants are perceived, are not as strong in Portland as they are in Seattle. 









%In an early analysis of this data, I attempted a dynamic analysis of \bat by fitting separate mixed-effects regression models at all 11 time points for which I have data. In other words, I augmented the analysis normally presented at the midpoint with identical procedures for the other 10 timepoints. The model summaries were presented in a table of coefficients like Table 2.3, where each row represents a time point and each column are the variables included in the model (cells are blank if that predictor was not significant). Interesting conclusions can be gleaned from this table, such as the changing effect duration has on the vowel over the course of its trajectory, or the interaction between sex and age being significant only in the second half of the vowel. While this analysis served as hacky way to do a dynamic analysis, it was flawed for two reasons. First, the odds of a Type I error increase substantially because of the many models required for such an analysis. More importantly though, the models were completely independent of each other and did not take into account measurements at neighboring timepoints. The measurements at any point in time are correlated with nearby timepoints, and this correlation is overlooked in the individual mixed-effects models. The various measures should be treated together in a single model, rather than the unrealistic separation assumed by the 11 linear mixed-effects models. The solution to this issue is to adopt more dynamic methodologies to the study of these vowels. 

%\begin{table}[htb]
%    \centering
%    \begin{tabular}{ l  rl r r   r r r   r r r }
%percent & \multicolumn{2}{c}{intercept} & duration & age:sex$=$Male %& age & sex$=$Male & style$=$reading \\
%0   &   $129$    & & & & \\
%0.1 &    $25.3$  & & & & \\
%0.2 & $-109$   & & & & \\
%0.3 & $-127$   & & & & \\
%0.4 & $-136$   & & & & \\
%0.5 & $-109$   & & & & \\
%0.6 &     8.34 & & & & \\
%0.7 &    87.8  & & & & \\
%0.8 &   112    & & & & \\
%0.9 &   142    & & & & \\
%1   &   158    & & & &
%    \end{tabular}
%    \caption{Model coefficients for 11 mixed-effects linear %regression models on \bat.}
%    \label{tab:previous_analysis}
%\end{table}