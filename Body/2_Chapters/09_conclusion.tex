\chapter{Conclusion}
\label{ch:conclusion}


\section{Overview}

In this study, I presented acoustic data on the front lax and low back vowels of 54 speakers from Cowlitz County, Washington. This is a region of the United States where English has not been analyzed sociolinguistically. In general, the West is an understudied region as far as dialectology is concerned, though the amount of research in the past five years suggests that this is rapidly changing. One of the primary findings in speech in the western states is the Elsewhere Shift, which appears to have come out of California and perhaps Canada more or less independently \citep{hinton_etal_1987, clarke_etal_1995}, and was triggered by the (near) merger of the low back vowels. The Elsewhere Shift has since been found in areas of the United States outside of California, including other western states and some parts of the Midwest \citep{fridland_etal_2016_pads, fridland_etal_2017_pads, strelluf_2019}. This study is the first to show that this shift has made its way into southwest Washington State.

Methodologically, this study combines traditional data collection techniques with innovative statistical modeling. The data was gathered using sociolinguistic interviews, which allowed me to immerse myself in the community and get a feel for its culture. But because there are more to vowels than single-point measurements, generalized additive mixed-effects models were used to analyze the formant dynamics and get a picture of the vowels' trajectories. This method proved useful as many of the findings in this study relate to the trajectories themselves, rather than absolute position in the vowel space. For example, in \bet and \ben, men did not change the position of their vowel with the women, but they did change the vowels' shapes with them; \bat lowered and retracted in this community, but the first half did so at a faster rate than the second half, resulting in a change from a U-shape to a pointed trajectory; and the overall spectral change in nearly every vowel gradually decreased in apparent time, meaning the vowels are becoming more monophthongal. I make heavy use of difference smooths to support these findings.

When analyzing this change though, I found that the relative timing of when the vowels shifted position is similar to that described by a pull chain. The low back vowels were consistently close, but unmerged, in this community, and I suggest that the approximation of the two was complete before the Silent Generation acquired language. Because \bat lowering and retraction occurs gradually across all generations---crucially with the women ahead of the men---I conclude that \bat began shifting at the latest by the 1930s, if not earlier. \bet followed suit with the Baby Boomers and then the Millennials began retracting \bit in the 1980s. I suggest that the structural relationship between the low back vowels and (preobstruent allophones of) the front lax vowels is supported. For the prenasal allophones, the relationship was not as clear, though the prenasal split in \ban was evident, particularly among the Millennials.

The timing of these changes correlated with cultural and demographic shifts in the area. When Long-Bell established its mills in the early 1920s, Longview was quickly populated with immigrants from across the country and the world. This mix of dialects may have been the reason for the low back merger, which then set the other vowels into motion. Some of the other shifts, particularly in the prenasal vowels, correlated with the fall of the timber industry and the resulting economic recession in the area in the 1980s. Younger speakers were suddenly in a different version of Cowlitz County that their parents and grandparents knew, and began orienting themselves towards Portland---and adopting the speech patterns found there.

This study contributes to the field in a variety of ways. First, it fills in a small gap in our current dialectological map of North American English. But it goes beyond a simple description; it further shows that speaker orientation towards or away from a particular place manifests itself in speech patterns. Furthermore, it provides the first extensive look at the formant trajectories of the vowels involved in the Elsewhere Shift, illuminated sources of variation that would not be easily found when vowel formants are sampled at only one point. This study also demonstrates the utility of generalized additive mixed-effect modeling on vowel formants and is among the first to do so in dialectology research.



\section{Limitations}

As with any study, this work is not without its limitations and it is useful to acknowledge shortcomings that may have an effect on the results.

First, the data presented here are entirely based on those collected via sociolinguistic interviews. While this generally provides a somewhat natural environment to collect speech styles, it is limited to that one style. I did collect more data from more formal tasks, such as a reading passage, a wordlist, and minimal pairs, but because the Elsewhere Shift was not among the features these tasks targeted, a limited number of tokens are available from these styles (and were ultimately discarded). And, because some of the vowel classes are relatively uncommon (such as \beng), additional tasks that focus on these allophones would be needed for a more robust analysis. Furthermore, recent research has adopted techniques to gather even more natural speech to collect stylized variants not found in sociolinguistic interviews \citep{vanhofwegen_2017_diss}, but this was out of the scope of this study.

Furthermore, while I did spend over a month in Cowlitz County, I was not able to do the sort of ethnographic fieldwork that produces detailed results on social meaning \citep{eckert_2000, hall_lew_2009_diss, pratt_2018_diss}. Many of the findings described here were confirmed with my extended family who live in the area, but ultimately, as an outsider, I cannot make firm claims regarding the culture of the area without additional fieldwork.

Regarding data collection, while I am thrilled to have collected 54 interviews (my goal was 30), it was just not enough to model age as a nonlinear predictor in the GAMMs. I could have included it as a linear effect, but I knew that sudden changes have happened in this area \citep{stanley_2018_pwpl} and I needed a model that could account for these types of changes. Ultimately, I chose generational divisions which I believe produced satisfactory results. Nevertheless, additional speakers of all ages would allow for a more continuous fit to the data, making it possible to analyze the nonlinear rate of change.

In the methods section, I mentioned that I did not ask participants for demographic information. I should have. I am reasonably confident that I got ages and genders correct, but additional demographic information should have been operationalized better. In particular, a survey that quantifies speakers' orientations towards or away from Cowlitz County or Portland would have been useful to model such views quantitatively. Of course, I was not aware of some of these cultural views until data collection had started (and sometimes not until transcription was complete). A return trip to the field to administer such surveys would be fruitful for this study.

Furthermore, the sample did not include very many minority groups. As explained in chapter \ref{ch:methodology}, there was one women who identified as half-Hispanic and another whose mother grew up on a Native American reservation. Only one person identified as homosexual. These minority groups are vastly overlooked in sociolinguistics generally, and while Cowlitz County is predominantly white, additional study on ethnic and other minority groups is necessary to fully understand English in Cowlitz County.

This study has only focused on general tendencies. Most visualizations were predicted values that grouped entire generations and sexes together. Some individual-level plots were presented to support the broader picture, but even then, trajectories were averaged for that speaker. A deep dive into the token level realizations of these vowels and speaker idiosyncracies would uncover a vast amount of variation that this overview overlooked.

Finally, in the analysis, I based my conclusions based on statistically significant differences. As mentioned in \ref{ch:methodology}, the idea of ``statistical significance'' has been recently questioned in the field of statistics \citep{wasserstein_etal_2019}. Though I was careful in interpreting the difference smooths, ultimately, I worked under the assumption that if something is statistically significant than it is also socially significant. (I believe this is a widespread assumption in quantitative sociophonetics.) I would like to see---as well as conduct---perceptual experimentation that tests emperically whether slight differences, such as those described in this study, are indeed perceptual and sociolinguistically charged.





\section{Future work} % i.e. "staking claims"

While this study has answered numerous questions about English in Washington, the Elsewhere Shift, and vowel trajectories, it has opened up additional questions. These are some of the directions I would like to explore.

First, I was only able to describe just a few of the vowels in this study. Every other vowel is potentially shifting in this community, based on findings from nearby areas. In particular, the back vowels \goose, \foot, and \goat are fronting in the West, and \face is monophthongal in Washington. Furthermore, there are other variants that I heard that I would like to explore, such as the offglide in \bash (\trap before [\textipa{S}]) in some of the older speakers and the monopthogization of \mouth, especially in the word \textit{Cowlitz}. It is unclear at this point whether a robust analysis of these vowels will yet be possible since they were not targeted linguistic variants as a part of the fieldwork.

There were a variety of other phonological features that I heard that would be potentially fruitful areas to explore.
\begin{itemize}
    \item There was variation in the amount of affrication in /\textipa{t}/ and /\textipa{d}/ before rhotics; \textit{train} or \textit{drain} were heard as [\textipa{t\*re\textsubarch{I}n}\textasciitilde\textipa{tS\*re\textsubarch{I}n}] and [\textipa{d\*re\textsubarch{I}n}\textasciitilde\textipa{dZ\*re\textsubarch{I}n}].
    \item It was very common to hear word-final /-\textipa{iN}/  realized with a high vowel but an alveolar nasal [\textipa{in}] rather than the more common [\textipa{IN}] or ``\textit{g}-dropped'' [\textipa{In}]. Impressionistically, some speakers had this as their majority variant.
    \item As expected, some older residents had the occasional intrusive \textit{r} in words like \textit{Washington} and \textit{wash}, and, in one older man, two tokens of \textit{watch}.
    \item There were sporatic instances of nonmainstream variables such as the merger of \north and \force, retention of /\textipa{\*w}/, insertion of [\textipa{t}] in [\textipa{ls}] clusters (\textit{Kel}[\textipa{t}]\textit{so}, \textit{el}[\textipa{t}]\textit{se}, \textit{all}[\textipa{t}]\textit{spice}), and the insertion of velar stops after utterance-final \textit{-ing}.
\end{itemize}
Furthermore, these speakers used variety of morphosyntactic features that were somewhat unexpected.
\begin{itemize}
    \item Many older speakers used invariable \textit{says}, as in ``I says to her\ldots'', and invariable \textit{was}, as in ``we was\ldots''. Some speakers also used \textit{come}, \textit{build}, \textit{ask}, and \textit{seen} in the past tense.
    \item Some middle-aged women used \textit{Do!} as a single-word-utterance imperative. None were caught on microphone, but my recollection is that as I was leaving a participant's house, I said, ``I think I'll go check out that park nearby,'' to which she responded, ``Do!''
    \item Intriguingly, two of the oldest men used \textit{I} first in a coordinated noun phrase, as in ``I and another guy'', ``I and a guy I drove truck with'', or ``I and her''.
    \item Many speakers used what are canonically past tense forms of verbs after auxiliary verbs, as in \textit{had took}, \textit{had did}, or \textit{we've came home}.
    \item There were miscellaneous other features such as double modals, \textit{liketa}, positive \textit{anymore}, \textit{needs washed}, and \textit{a couple three}.
\end{itemize}
Few of these are frequent enough for any robust analysis, and some of them would be difficult to elicit. Future work may benefit from targeting some of these marginal phonological and morphosyntactic processes in Cowlitz County and elsewhere in the West.

Many of the proposed sound changes in this study can be confirmed with legacy recordings from speakers before the Silent Generation. For example, I propose that the low back merger was complete by 1930 and that \trap retraction began around that time. Fortunately, I have already collected several hours of such legacy recordings, courtesy of David Wilma and the Cowlitz County Historical Museum. I have not yet analyzed this audio, but one recording from the 1950s contains a 20-minute oral narrative by a man born in Cowlitz County in 1886 (whose mother was also born in the area in the 1850s). I hope to analyze this audio in the future to get a complete picture of Cowlitz County. These recordings would also shed light on other vowel shifts that were less clear, like the retraction of \bin and the very fronted \ben among the older women in this sample.

Most of the linguistic changes proposed in this study involved changes in vowel trajectories. Presumably, vowel trajectories can index particular social meanings, just as midpoints can. Sociolinguistic and psycholinguistic perception studies, particularly using synthesized speech, would provide evidence to support this hypothesis. Additionally, such studies would help understand (and operationalize) the social meaning embedded into these variants. Finally, \lot and \thought were differentiated primarily by vowel trajectory rather than position. A follow-up perception study would aid in understanding how much trajectory change is needed for the vowels to be perceived as different.

Finally, many of the cultural shifts that I have described here may apply to other areas in Southwest Washington and additional work on areas near Cowlitz County help to understand the extent of these shifts. In particular, Wahkiakum County is immediately west of Cowlitz County and appears to be more rural and even more dependent on the timber industry. Did the changes in the 1970s affect them in the same way? Further south in Clarke County, Vancouverites are part of the Portland Metropolitan Area; do they exhibit any Washington features in their speech? Additional work in these nearby areas may aid in understanding how the Elsewhere Shift diffuses across the region. Was it brought over by immigrants to the area or was it an internal development? Furthermore, work in these nearby areas would help identify whether some of the anomalous speech patterns, like \ben going in the opposite direction from \ban and \bin, are more widespread than just the sample studied here.





\section{Conclusion}

As the editors of \textit{Speech in the Western States} put it, there are still ``there be dragons'' areas of our dialect maps of the United States \citep[172]{fridland_etal_2017_pads}. This study has filled in one small portion of those maps, southwest Washington, and has uncovered additional variation in Washington State. Meanwhile, it has shown that there is variation and change in vowel trajectories in the West.
