\chapter{Western American English}
\label{ch:lit_review}




\section{Introduction}
\label{sec:lit_review_intro}

In the spring of 1986, Leanne Hinton led a graduate seminar at the University of California, Berkeley to study the pronunciation of English in California at that time. They noted that earlier studies on California English described the variety as relatively unremarkable, lacking distinctive features of its own. For example, Allan Metcalf, in a report on the \textit{Linguistic Atlas of the Pacific Coast} (\textit{LAPC}), says,
\begin{quote}
    The pronunciation of English in California and Nevada is unobtrusive, a bland blend of patterns found in the north and midlands in the eastern United States. To the linguist as well as to the untrained ear, it most often seems to be an American English `shorn of all local peculiarities' [\citep[192]{pei_1967}]---like the dog in the Sherlock Holmes adventure of Silver Blaze, notable for not being noticed \citep[8]{metcalf_nd}.
\end{quote}
However, in parodied imitations of Californians in the media, Hinton and her students noticed exaggerated phonetic features that were not found in early phonetic descriptions from the area. Had the language of California changed? The goal of the seminar was to compare their findings to the 270 Californians interviewed in the 1950s as part of the \textit{Linguistic Atlas of the Pacific Coast}.

Their results, eventually published as Hinton, Moonwomon, Bremner, Luthin, Van Clay, Lerner, \& Corcoran \citeyearpar{hinton_etal_1987}, became a pivotal study in speech in the West because they were perhaps the first to document the lowering and backing of \kit, \dress, and \trap in California. They found that younger, white, urban speakers tended to exhibit these patterns the most and proposed that these changes were the beginnings of a new shift in California English: ``[i]t is quite possible, then, that these new sound shifts will progress along the lines of many other California phenomena, becoming more extreme and spreading geographically'' \citep[126]{hinton_etal_1987}. While other studies have shown that these patterns have indeed spread across the West (cf. \citealt{fridland_etal_2016_pads}; \citealt{fridland_etal_2017_pads} inter alia), in this chapter I will show that they have spread geographically into Cowlitz County, Washington, providing the first conclusive evidence that the shift can be found in the speech of Washingtonians.

\subsection{A note on terminology}
\label{sec:terminology}

The vowel shift described by \citet{hinton_etal_1987}---the lowering and retraction of the front lax vowels \bit, \dress, and \bat---now goes by many names. In this study, I refer to it as the \textit{Elsewhere Shift}\footnote{\citet[20]{strelluf_2019} points out that the term was used ``somewhat jokingly'' at NWAV in 2016. I was not present at that conference, so I missed that connotation. Nevertheless, I will continue to use this term.}. As justification for using this term, this section explains the other names that have been used and why \textit{the Elsewhere Shift} was selected as the most appropriate for this study.

The most common terms for this vowel pattern describe where in North America it can be heard. One of the most popular names is the \textit{(Northern) California Vowel Shift}, coined by \citet{eckert_2008} because the shift has primarily been documented in the speech of Californians (cf. \citealt{hall_lew_etal_2015, janoff_2018, podesva_2011, podesva_etal_2015, villarreal_2016_pads, villarreal_2018} and many others). However, in light of recent research showing the presence of these changes in Nevada \citep{fridland_kendall_2017_pads}, Oregon \citep{conn_2000_diss, nelson_2011, becker_etal_2016_pads, mclarty_etal_2016}, Colorado \citep{holland_brandenburg_2017_pads, holland_2019}, Arizona \citep{hall_lew_etal_2017}, and New Mexico \citep{brumbaugh_koops_2017_pads}, the editors of the \textit{Speech in the Western States} volumes \citep{fridland_etal_2016_pads, fridland_etal_2017_pads} propose that the label \textit{Western Vowel Pattern}\footnote{Actually, as early as 2004, the term \textit{Western Vowel Shift} was used to describe this pattern in Arizona \citep{hall_lew_2004}.} be used. Meanwhile, because of its presence across most of Canada, the term \textit{Canadian Shift} has been used as well  (\citealt{clarke_etal_1995, boberg_2005, sadlier_brown_tamminga_2008, roeder_jarmasz_2010, kettig_2014} and many others) because they are ``talking about Canadians'' \citep{li_etal_2018}. These differences in terminology also reflect the trend that research in California and Canada has progressed more or less independently. Furthermore, when Californians and Canadians are compared directly, there are slight differences \citep{kennedy_grama_2012, hagiwara_2006} leading some to argue for the need to differentiate the two patterns.\footnote{For example, \citet[49]{kennedy_grama_2012} use the benchmarks provided in the \textit{Atlas of North American English} to define whether a speaker’s vowels are shifted. Most of their sample lowers \kit, \dress, and \trap past the threshold that defines the Canadian Shift, but their tokens of \lot cluster around the threshold. They conclude that because the front vowels were lowering while \lot was not sufficiently backed, ``it suggests that the California Shift is a different phenomenon from the Canadian Shift.'' However, based on the Short Front Vowel Shift Index (see \S\ref{how_shifting_is_measured}), which does not consider the low vowel(s), \citet[21]{boberg_2019} states that the shifts in California and Canada are, ``for all intents and purposes, the same thing.''}

A few of the other proposed labels for this vowel pattern are more descriptive of the vowels themselves, rather than the geographic regions involved. For example, \citet{hickey_2018} uses the term \textit{Short Front Vowel Lowering}. And Boberg \citeyearpar{boberg_2019} uses a similar term, the \textit{North American Short Front Vowel Shift}, as opposed to the \textit{New Zealand Short Front Vowel Shift}, in which the vowels move in the opposite direction from what is described here. Most recently, the term \textit{Low-Back-Merger Shift} has been proposed \citep{becker_2019_pads}, which makes the controversial theoretical stance that the shift is triggered by the low back merger.

Finally, there are labels in circulation that make no direct reference to geographic regions or vowels. \citet{labov_1991} may have been the first to assign a name to parts of to this pattern, the Third Dialect.\footnote{\textit{Third Dialect} more formally refers varieties that have the low back merger and /\textipa{\ae}/ being realized as a low front vowel, except before nasals where it is raised \citep[30]{labov_1991}. Given that bulk of that paper was written in 1980 (p. 34) and that the shifting in the front lax vowels in Third Dialect regions has only been documented since then, it is unclear whether the label should be applied to the lowering and retraction of front vowels, even if they are a consequence of the low back merger.} This is useful for researchers who study both the Pacific Coast and Canada \citep{swan_2018} or neither region \citep{durian_2012_diss}. The term \textit{Elsewhere Shift} has also been proposed to serve this purpose, with \textit{elsewhere} presumably refering to varieties that do not participate in the Southern Vowel Shift or the Northern Cities Shift, though this is relaxed somewhat due to the possible influence of both the Northern Cities Shift and the Elsewhere Shift in the same region \citep{mason_2018}. The term \textit{elsewhere} may also refer to the elsewhere allophones of the front lax vowels, which are the ones that are involved in this shift. The term has not gained very much popularity, but the dual meaning of geographic and phonological distribution is appealing.

For this dissertation I have opted to use this term, the \textit{Elsewhere Shift}. Terms that refer to geographic regions (\textit{California Vowel Shift}, \textit{Canadian Shift}, \textit{Western Vowel Pattern}) inadvertently exclude areas outside of some region where the pattern can be found. The terms \textit{Short Front Vowel Lowering/Shift} are descriptive enough when only \kit, \dress and \trap are included for study, but since the low back merger may be related to this lowering, at least in Western American varieties (see \S\ref{sec:structure_of_elsewhere_shift}), I argue that the label does not fully capture the shift. Thus, the geographic and vowel-ambiguous terms \textit{Elsewhere Shift} and \textit{Third Dialect} appear to be most appropriate presently for this study.

Another reason to use the term \textit{Elsewhere Shift} is because it appears to apply to the elsewhere allophones of the front lax vowel phonemes. For example, the \trap vowel can be divided into many different allophones, including prelateral, prerhotic, prenasal, pre-/\textipa{N}/, and pre-/\textipa{g}/\footnote{These allophones and their labels are explained in detail in \S\ref{word_classes}.}. But the elsewhere allophone, that is \trap before obstruents (except /\textipa{g}/), is the one that appears to be mostly consistently affected by the Elsewhere Shift. Similar patterns can be found with \dress and \kit. Therefore, the term \textit{Elsewhere Shift} appears to apply appropriately to geographic regions not part of the Northern or Southern Shift as well as the allophones not otherwise involved in some other phenomenon.

Finally, what vowels are included in the Elsewhere Shift? Some authors consider all Western features to be a part of the shift, including the lowering and retraction of front lax vowels, the low back merger, and the fronting of back vowels \citep{eckert_2004, eckert_2011, podesva_2011, podesva_etal_2015, donofrio_etal_2017_pads, donofrio_etal_2019}. Meanwhile, Boberg takes the stance that back vowel fronting occurs ``coincidentally, rather than causally'' with the shifting front lax vowels \citet[12]{boberg_2019}. For the purposes of this dissertation, I will use the term the \textit{Elsewhere Shift} to refer only to the front lax vowels as distinct shifts from the low back merger. The fronting of back vowels will not be considered in this study and thus is not a part of the definition of the term as I use it here.




\section{Geographic distribution of the Elsewhere Shift}
\label{sec:geography_of_elsewhere_shift}

Since that graduate seminar in Berkeley, numerous studies have documented the Elsewhere Shift in many parts of English-speaking North America. The low back merger specifically has been documented extensively in California \citep{moonwomon_1991_diss, hagiwara_2005, holland_2014_diss} and to a lesser degree in New Mexico \citep{brumbaugh_koops_2017_pads} and Montana \citep[122]{bar_el_etal_2017}. In Oregon, \citet{mclarty_etal_2016} find that both younger speakers and older archival speakers have a high degree of overlap between the two vowels, suggesting that the merger has been present for several generations. Similarly, the merger is found in Washington, even among the oldest speakers today \citep{wassink_2015, wassink_2016_pads}. In some studies, the two vowels are assumed to be merged without further commentary \citep{eckert_2008, podesva_2011, kennedy_grama_2012}.

As for the front lax vowels, the \textit{Atlas of North American English} actually did not find shifting in the West, stating that ``the West’s means for the short vowels\ldots do not stand out from the others, but are found slightly below the center of the main distribution. The West does not participate strongly in the Canadian Shift'' \citep[284--285]{labov_2006}. However, more focused studies with recent data have found evidence to support \citet{hinton_etal_1987}. For example, just focusing on \trap retraction, \citet[16]{donofrio_etal_2017_pads} find that younger people use increasingly retracted variants, especially the women, in California's Central Valley. \citet{kennedy_grama_2012} show that all speakers under the age of 30 in their sample lowered \trap, \dress and \kit, and \citet{holland_2014_diss} provides evidence that women and younger speakers have lower or backer variants of all three vowels, suggesting a change in apparent time. Similar patterns were found with respect to each of these vowels in San Francisco \citep{hall_lew_etal_2015, cardoso_etal_2016_pads}, Santa Barbara \citep{janoff_2018}, and elsewhere in California \citep{brotherton_etal_2019}. Thus, it is apparent that these vowels are indeed lowering in California.

In addition to studies focusing on California English, researchers further north into Oregon have likewise documented the Elsewhere Shift, though the patterns are less clearly defined. With speakers based primarily in the Southern Willamette Valley, the area closest to California, \citet{nelson_2011} finds that younger speakers are lowering \kit, backing \dress, and lowering and backing \trap. Further north into Portland, \trap is certainly retracting and lowering, but it is unclear whether \dress and \kit are too \citep{conn_2000_diss, becker_etal_2013}. Recently, \citet[116--118]{becker_etal_2016_pads}  have found that 74\% of their sample retracted \trap, with age as a significant predictor but not gender. However, they find that only a third of their speakers lower \dress, with women participating more in this shift. Even fewer people (just three in their sample of 34) had \kit lowering. \citeauthor{becker_etal_2016_pads} therefore propose an implicational hierarchy: \kit lowering implies \dress lowering which implies \trap retraction. However, in a study of the shift over real and apparent time in Oregon, \citet{mclarty_etal_2016} find that both younger and older speakers shift all three vowels, though the older speakers do so to a lesser degree. However speakers from archival recordings have very little evidence of the shift. While there may be differences in the minutia of the shift, these Oregon-based studies all point to the idea that the Elsewhere Shift is less uniform than it is in California and that it has developed slowly over several generations with younger speakers shifting more vowels and to a greater degree than older speakers. Furthermore, it appears that the lowering of lax vowels in Oregon is a pull shift since \trap is the most advanced, followed by \dress, and then \kit.

While researchers were documenting this vowel shift along the Pacific Coast, the same patterns were independently being found in all parts of Canada. \citeauthor{clarke_etal_1995} find it in speakers primarily based in Toronto and say that it is ``superficially\ldots virtually identical'' with the shifts in California \citeyearpar[213]{clarke_etal_1995}. In Toronto, the shift is equally robust across ethnicities \citep{hoffman_2010} and in other parts of Ontario it is almost as advanced as it is in the capital \citep{roeder_2012} though it is nearing completion \citep{roeder_jarmasz_2010}. Towards the east, early reports in Ottawa find that retraction of \trap is being led by younger women (\citealt[151--153]{woods_1979}; \citeyear{woods_1993}, \citealt{de_wolf_1992}), which is confirmed, together with \dress and \kit retraction, in more recent data \citep{boberg_2005}. Even though \textit{Atlas of North American English} reports that the Canadian Shift has not spread as far eastward as the Atlantic Coast \citep[220]{labov_ash_boberg_2006_anae}, the shift is found to be active in St. John's, Newfoundland (\citealt{hollett_2006},  \citealt{darcy_2005}, see also \citealt{clarke_1991}) and Hallifax, Nova Scotia (\citealt{sadlier_brown_tamminga_2008}; see also \citealt{boberg_2008}). Though it is less advanced than in other parts of Canada, the shift is also present in the Canadian Prairies \citep{boberg_2011, hagiwara_2006}. Finally, it is found to be most advanced in Vancouver \citep{hall_2000, tamminga_sadlier_brown_2008, roeder_etal_2018}, approximately equally across ethnicities \citep{presnyakova_etal_2018}, with women about a generation ahead of men \citep{esling_warkentyne_1993}. It is clear then that the Elsewhere Shift is as widespread and vigorous in Canada as it is in California and Oregon, especially in British Columbia. However, the two have been treated as ``independently occurring phenomen[a]'' \citep[41]{kennedy_grama_2012} in the majority of studies, with \citet{boberg_2019} and the work by Julia Swan (see below) as some of the few exceptions linking the two.

Conspicuously absent from these many regions where the Elsewhere Shift can be found is Washington. Despite pressure from California and Oregon to the south and from British Columbia to the north, Washington has appeared to resist the ever-reaching influence of the Elsewhere Shift: ``[i]t is curious that Canadian and California English should display such a similar trend while not being geographically contiguous neighbors of each other, since there is currently no evidence documenting the same type of shift in the geographic space between them'' \citep[30--31]{swan_2016_diss}. Alicia Wassink has stated that the Elsewhere Shift is not present in Washington: ``Seattle Caucasians do not participate in the retraction of /\textipa{\ae}/ \bat and /\textipa{E}/ \bet\ldots Additionally, we do not see the lowering of the /\textipa{I}/ \bit and /\textipa{E}/ \bet vowels'' \citeyearpar[84]{wassink_2016_pads}. She posits that Washingtonians' lack of participation in the Elsewhere Shift ``may be functioning as an important marker, distinguishing subregions in the West'' \citep[53]{wassink_2015}.\footnote{This lack of the Elsewhere Shift in Washington strengthens the argument I make in chapter \ref{ch:discussion}, which is that younger speakers orient themselves more towards Portland rather than to other Washingtonians.}, I suggest that the presence of the shift in Cowlitz County Washington is a large state with relatively few linguistic studies focused on its residents, so more work is needed to accurately describe the presence or absence of the shift in all parts of the state. But if the Elsewhere Shift is to be found in Washington, it would bridge the gap between California and Canada, perhaps finally uniting the two linguistic phenomena as a Pan--North American sound change. Or, if differences between the California and Canadian vowel shifts persist, then its manifestation in Washington would provide for an intriguing case of competing---albeit very similar---linguistic features.

However, one reason for why relatively little is known about the Elsewhere Shift in Washington may have to do with the history of dialectology in the state. As described in Chapter \ref{ch:introduction}, there was relatively little research on English in Washington before ten years ago, possibly because the speech in that area was considered to be devoid of regional characteristics, even more so than California. However, when Alicia Wassink and her colleagues introduced prevelar raising as a feature of Pacific Northwest English in 2009\footnote{Prevelar raising is the raising of \trap before /\textipa{g}/, so that words like \textit{bag}, \textit{flag}, or \textit{dragon} are pronounced with [\textipa{E}] or even [\textipa{e:}]. It was already known to be a part of Wisconsin English, but as far as I can tell, \citet{wassink_etal_2009} were the first to describe this feature in the Pacific Northwest. It has later been described in many regions of North America \citep{stanley_2019_BEG_paper}, but it is one thing that makes Washington stand out among the western states \citep{wassink_2016_pads}.}---especially since there were only isolated occurrences of it in the \textit{Linguistic Atlas of the Pacific Northwest} (i.e. Reed \citeyear{reed_1956}, \citeyear{reed_1961})---that set the tone for subsequent research questions for the next decade in Washington. Ever since that presentation, the majority of acoustic research on Washington English has focused on prevelar vowels and their distribution across regions, ages, genders, ethnicities, and other ideologically-based social groups (\citealt{wassink_2011, wassink_2015, wassink_2016_pads, freeman_2014, riebold_2015_diss, swan_2016_proceedings, stanley_2018_pwpl} and many others). To my knowledge, very few studies have looked at the front lax vowels in Washington outside of the prevelar environment.

One exception is Julia Swan’s research, which has focused on the direct comparison of English in Seattle, Washington and Vancouver, British Columbia. Though she, too, primarily describes differences in the realization of prevelar vowels, \citet[8]{swan_2016_proceedings} finds that retraction of \trap before fricatives (as opposed to stops) is more advanced for Vancouver speakers than it is for Seattle speakers, a pattern described in Canada by \citet[214]{clarke_etal_1995} and \citet{boberg_2019}. Furthermore both groups have nearly identical trajectories for pre-/\textipa{d}/ tokens of \trap \citeyearpar[10]{swan_2016_proceedings}, and F2 measurements were not statistically significantly different from each other in pre-obstruent environments \citep{swan_2015}. Given that Vancouver is the part of Canada where the Elsewhere Shift is most advanced \citep{hall_2000, tamminga_sadlier_brown_2008, roeder_etal_2018}, Swan indirectly reports that \trap retraction may be found in Seattle.

In summary, the Elsewhere Shift can be found in a very large geographic area of North America. It extends across all of Canada, and along the Pacific Coast from Southern California to at least as far north as Portland. It can even be found in areas not traditionally part of Third Dialect regions such as Hawaii \citep{grama_etal_2012, kirtley_etal_2016}, Alaska \citep{bowie_etal_2012}, Ohio (\citealt{durian_2012_diss}; \citealt[20]{thomas_2001}), Illinois \citep{bigham_2010}, Michigan \citep{nesbitt_mason_2016, mason_2018}, Texas \citep[20--21]{thomas_2001}, Massachusetts \citep{stanford_etal_2019}, and Georgia \citep{stanley_2019_LCUGA6}. If it is the case that speakers in Washington are clinging to traditional variants, we have a noteworthy case of resistance to such a widespread change, which may be grounded in strong opposition to the ideological personae expressed in these variants. However, as this study reports, many speakers in Cowlitz County \textit{do} have the Elsewhere Shift in their speech, meaning that they are participating in the macro-level changes of the region. In other words, they are distinguishing themselves from Seattleites. These findings provide some evidence against the claim that Washington is resisting the change and suggests that the shift has crossed the border into Washington.




\section{A structural description of the Elsewhere Shift}
\label{sec:structure_of_elsewhere_shift}

As a consequence of this large amount of research on front lax and low back vowels in North American English, we have learned a great deal about the structure of this shift. However, the degree to which vowels shift varies across regions and from study to study, and many questions remain regarding the structural relationship between the front lax vowels and their connection to other shifting vowels.

\subsection{The position of the low back vowel(s)}
\label{position_low_back_vowels}

The most defining feature of Western American English is the low back merger \citep[277]{labov_ash_boberg_2006_anae} and has been reported in numerous communities. As far as how the two vowels are merging, there are different reports of this process. In the West, it has been found that \thought lowers and fronts to merge with \lot \citep{hall_lew_2013}. In Utah, just the opposite was found: \thought was remarkably stable in real time, and it may have been \lot that backed to merge with \thought to result in a backed merged vowel \citep{bowie_2017_pads}. \citet{donofrio_etal_2017_pads} report a similar pattern in California's Central Valley. Most famously, \citet{herold_1990_diss} proposes a merger by expansion in which the distinction between the two is simply lost, and speakers realize tokens anywhere in the combined vowel space of the two historical vowels.

Regarding the relative position of the merged vowel, there is variation across studies. \citet{holland_brandenburg_2017_pads} find that the F2 of the merged low back vowel is decreasing in apparent time, suggesting that the vowel is getting more backed. Furthermore, \citet[23]{donofrio_etal_2017_pads} report that in Redding, California, the two vowels merged first, and then the now-merged vowel raises to a position that is higher than most other regions in the United States, creating a triangular vowel space with \bat as the lowest vowel; in Bakersfield and Merced, this higher merged vowel was achieved by \lot raising to meet the stably high \thought. This raising of the merged vowel, accompanied with \bat-retraction and \bet- and \bit-lowering creates an elegant description of a rotated vowel space as a result of the Elsewhere Shift. There are some exceptions (such as the relatively fronted merged vowel in Washington reported by \citealt{wassink_2016_pads}), but the general tendency is for the merged vowel to be backed and possibly raised.

However, what appears to be a more common finding in studies in the West is that speakers are on their way towards merging the two vowels. For example, \citet{moonwomon_1991_diss} analyzes the two vowels in a variety of environments and shows that the oldest speakers retain the distinction except before nasals and fricatives while the younger speakers all have a merger or a partial merger in all environments. \citet[367]{hall_lew_2013} reports that Chinese Americans had a more advanced merger, but it was not complete in San Francisco in 2008--2009. In Colorado, the two vowels were close, but \lot was consistently more fronted than \thought, especially for the men, suggesting a near, but so far incomplete, merger \citep{holland_brandenburg_2017_pads}. In Nevada, \thought is further back in the vowel space, but women are closing the gap \citep{fridland_kendall_2017_pads}. Most notably, \citep{dipaolo_1992} finds that \lot and \thought are distinct in Salt Lake City, despite other reports of merger in the region. Close to Cowlitz County, \citet{becker_etal_2016_pads} reports that nearly 40\% of their Portland-based sample retain the distinction.

These various studies point out that despite being a widespread feature of the West, there is a fair amount of variation. In some areas, the vowel is reported to be completely merged. However, there are pockets where the data suggests more of a near merger. In some areas, one vowel is stable in apparent time, with the other shifting towards it. The merged vowel is reported to be somewhat fronted, relatively backed, or backed and raised. However, in nearly every case, if the low back merger is not complete, it is on its way towards completion.

\subsection{The relationship between the front lax vowels}
\label{sec:relationship_between_front_lax_vowels}

In some ways, because \trap retraction occurs primarily in areas that have the low back merger, it is reasonable to propose that the merger of \lot and \thought was the start of a chain shift. This low back merger could have caused \trap to lower and retract to fill the void left by \lot, which in turn caused \dress and \kit to shift. In fact, \citet[139]{gordon_2006} proposes this very idea, that this merger is the underlying cause of the lowering and retraction of front lax vowels. He is supported by \citeauthor{clarke_etal_1995} who suggest that ``it is precisely the merger of the cot/caught vowel which serves as the pivot of the Canadian Shift'' (\citeyear[212]{clarke_etal_1995}; see also \citealt{dedecker_mackenzie_2000}; \citealt[220]{labov_ash_boberg_2006_anae}, \citealt{boberg_2019}). In an approach driven more by phonological theory, \citet{roeder_gardner_2013} propose that the merging of \lot, \thought, and \palm caused the restructuring of \trap, which in turn led to its retraction. A language-internal motivation for the shift accounts for the similarities between the Elsewhere Shift in California and Canada because migration patterns cannot do so. It also fits with a principle of chain shifts that Labov has proposed: ``lax nuclei fall along a nonperipheral track'' \citeyearpar[194]{labov_1994}. In theory, the chain shift hypothesis is an elegant explanation for the Elsewhere Shift.

However, phonetic data sometimes fails to support the hypothesis of a chain shift. First, it is unclear whether the low back merger is indeed the trigger. In Illinois, \citet{bigham_2010} finds the correlation between the low back merger and \trap retraction at a community level, but when examining individuals there were some speakers with the merger and no backing of \trap while others did not have the merger yet still had a backed \trap. In California, \citet[51]{kennedy_grama_2012} show that the merged low back vowel is not retracted as it is in Canadian speakers, but their speakers do have a lower and more centralized \trap. They take this to mean that the Elsewhere Shift may not necessarily have been triggered by the retraction of the low back vowel, at least in California \citep[see also][]{grama_kennedy_2018}. In fact, \citet[121]{holland_2014_diss} finds evidence against the low back merger being the trigger, and instead finds that---if the Elsewhere Shift is a chain shift at all---that the retraction of \dress may have been the trigger \citep[see also]{holland_2019}. Essentially, we find that each community appears to have different realizations of these vowels, and that a single explanation simply does not describe the phonetic patterns found in all areas with the Elsewhere Shift.

Furthermore, it is unclear whether the retraction of front lax vowels happened one at a time, as a chain shift would predict, or all in parallel. Some studies along the Pacific Coast find that the shifting of one vowel implies the shifting of another and propose an implicational hierarchy bearing the resemblance of a chain shift (\citealt[116--117]{becker_etal_2016_pads}; \citealt[121]{holland_2014_diss}). But others find parallel movement across generations with no indication that any one vowel moved first \citep{pratt_etal_2018, donofrio_etal_2019}. Furthermore, evidence from rural Ontario \citep{lawrance_2002_thesis} and Montreal \citep{boberg_2005} also suggest parallel movement of all three vowels. This discrepancy between how the Elsewhere Shift progresses over time further suggests that no one explanation can fully explain the phonetic data found in various communities.

% TODO: Renwick: Q. Is it really the case that vowels must be kept maximally distinct from one another? Is that Labov's position, or your position? Might there be particular models of vowel inventories that do or don't support this position? (Could you take a broader reach into the phonetic literature/motivations for this phenomenon, if it is real?)
Another possible explanation involves a combination of a chain shift and parallel changes. Chain shifts affect more than one vowel by means of a cause-and-effect relationship \citep[119--121]{labov_1994} and the underlying force driving these shifts is the need to keep vowels maximally distinct from one another.\footnote{Across the world's languages, vowels tend to position themselves around the edges of the F1-F2 space \citep[228,285]{ladefoged_johnson_2011}.}. While vowels may not be maximally distinct in the sense that they employ a host of secondary articulations and suprasegentals, at the very least, they need to be sufficiently perceptually distinct from one other to remain contrastive.). The movement of one vowel to fill the void left by another can account for the low back merger, \trap retraction, and the lowering of \dress and \kit, but it does not necessarily explain the retraction of \dress and \kit. Instead, the two higher vowels may simply move by analogy to the retraction of \trap (\citealt[232]{durian_2012_diss}; \citealt[151]{boberg_2005}), similar to back vowel fronting. This joint explanation has some merit because it appears to explain the patterns found in communities where the strict chain shift or strict parallel shift fails.

Finally, there are conflicting reports of whether the vowels retract (suggesting a lowered F2), lower (suggesting a higher F1), or both. Regarding the shift in Canada, \citet{boberg_2005} points out that while the difference may seem superficial, it has major implications for the structural nature of the shift. If \trap is retracting and \dress and \kit are lowering, that suggests a rotation in the vowel space that would be characteristic of a chain shift. However, if (as he finds in Montreal) the lax vowels primarily retract and retain the same height, this would be better described as a series of parallel shifts, akin to what is found with back vowel fronting in many varieties of American English and therefore not a chain shift. It is also possible that one pattern may be found in some varieties and another pattern is found in others.

In some cases, retraction and lowering are both reported, but they occur at different times within the same community. \citet[144]{boberg_2005} found lowering of \trap between the oldest and middle generations and then retraction of \trap and \dress between the middle and younger generations. Conversely, \citet{donofrio_etal_2019} find that \trap retraction occurred first in California and it was only the Millennials that have lowered it. In Lauren Hall-Lew's sample of residents of the Sunset District in San Francisco, younger participants and women produced a backer vowel, especially in the word list style (as opposed to the reading passage), where \bat was significantly lower as well \citep{hall_lew_etal_2015}. An analysis of these same speakers' conversation style found that similar results in regards to sex and age, only the primary dimension of change was height, not backness \citep{cardoso_etal_2016_pads}. All this goes to show that different processes can result in the same eventual outcome.







\section{The prenasal split}
\label{sec:prenasal_split}

The Elsewhere Shift applies to the elsewhere allophones of \trap, \dress, and \kit. However, in some cases, prenasal allophones of these vowels raise instead of lower, a phenomenon called the prenasal split. In other words, the distance between the prenasal and elsewhere allophones increases due to their movement in opposite directions. Thus, a full treatment of the Elsewhere Shift would be incomplete without a description of the variation found in this environment as well.

There are clear articulatory, acoustic, and perceptual reasons for why vowels in prenasal environment are raised. The articulation of a nasal consonant requires the opening of the velum to allow airflow through the nasal cavity. Because the velum is a relatively slow-moving articulator, anticipatory coarticulation occurs during the vowel in preparation for the following nasal. Therefore, for a portion of the duration of the vowel, air flows out both the mouth and nose, producing a nasalized vowel. Acoustically, a nasalized vowel differs from an oral vowel in various ways, including the presence of a strong concentration of energy in F1 region of frequencies called the nasal formant, which is the result of the nasal cavity being used as a resonating chamber. In a spectrogram, this nasal formant appears distinct from F1 of high and mid vowels, while it is indistinguishable from the F1 of low vowels other than widening its bandwidth \citep[193--194]{olive_etal_1993}. This extra nasal formant, particularly with \trap, causes vowels to be perceived as higher \citet{wright_1975}. This perception can be then be exploited by speakers to produce a higher vowel: \citet{dedecker_nycz_2012} show that some New Jersey speakers use tongue position to produce a tensed [\textipa{\ae}] while other use nasalization to simulate the auditory effect. \citet[334]{mielke_etal_2017} state that ``the most obvious explanation for the development of /\textipa{\ae}/ raising before nasals is that the acoustic effects of nasalization were transphonologized to tongue position.'' In other words, the side effects of the anticipatory coarticulation of the following nasal consonant are exaggerated onto the vowel itself, creating a raised variant that cannot be fully explained by phonetic effects alone. The \textit{Atlas of North American English} states that ``there is no doubt that nasal allophony has been translated to the phonological level,'' \citep[175]{labov_ash_boberg_2006_anae}  further supporting additional raising beyond coarticulatory effects.

\ban-raising specifically is well-attested in varieties with and without the Elsewhere Shift. Prenasal tokens were the most raised allophone of \trap in San Francisco \citep[43]{cardoso_etal_2016_pads} and Oregon \citep{becker_etal_2016_pads}. In fact, Thomas finds that prenasal raising ``occurs widely in North American English'' and that it ``appears to be largely a twentieth-century phenomenon'' \citeyearpar[52]{thomas_2001}. Its frequency in the West appears to be conditioned by ethnicity. Chinese Americans \citep[43]{cardoso_etal_2016_pads} and speakers of Chicano English \citep[34]{eckert_2008} have been shown to have less raising of \ban while Spanish speakers have a greater separation between \bat and \ban than California university students \citep{holland_2014_diss}.

Some researchers have described the specific realization of \ban in detail. It is transcribed with a central offglide, such as [\textipa{me\super @n}] \textit{man} \citep[34]{eckert_2008} or with a raised diacritic [\textipa{\|'\ae}] \citep[132]{gordon_2006}. When comparing trajectories of Seattle and Vancouver speakers, \citet{swan_2016_proceedings} describes the trajectory of \ban in Seattle as starting high and front and lowering and dramatically retracting over the course of its duration; in Vancouver it raises and backs gradually along its duration. These realizations are slightly different from the rising-falling pattern found by \citet{mielke_etal_2017}, which peaks just before the midpoint of the vowel. \citet{brotherton_etal_2019} find that secondary features like diphthongization and nasalization play important roles in differentiating \ban from \bat in the prenasal split. All these studies show that speakers with a greater prenasal split tend to have more diphthongization in \ban, suggesting that more trajectory-based research on this vowel is needed.

What is less clear is the extent to which \ben and \bin are raised. \citet[106--107]{holland_2014_diss} found evidence of \ben-raising in apparent time in California, but women’s \ben was lower and Spanish speakers’ was fronter. In a primarily Toronto-based sample, \citet{dedecker_mackenzie_2000} found that \ben and \bin were lowered less than in other environments, but this does not necessarily imply raising. In San Francisco, \bin was not any different from \kit and \ben actually retracted in apparent time at the same rate as other allophones of \dress, leading \citeauthor{cardoso_etal_2016_pads} to the conclusion that the nasal split does not apply to \dress \citep[43--44]{cardoso_etal_2016_pads}. Furthermore, there are some scattered reports of the \textit{pin-pen} merger in the West, such as in Riverside, California \citep[31]{metcalf_1972}, Trinity County, California \citep{geenberg_2014_diss}, Bakersfield, California \citep{warren_fulop_2014}, older Utahns \citep{lillie_1998_thesis}, and possibly Seattle \citep{scanlon_wassink_2010}. Thus, a hypothetical claim that all three front lax vowels raising before nasals is untenable based on research in West. It appears that the prenasal split applies chiefly to \trap, though further investigation on \ben and \bin is needed to understand what regional patterns there may be, if any.

Of particular interest is the velar nasal and its effect on the front lax vowels (\bang, \beng, \bing). Prevelar raising is known to affect vowels before /\textipa{g}/ (\bag, \beg, \vague) in the Pacific Northwest, the Upper Midwest, and Canada \citep{stanley_2019_ADS}, but voiced velars (that is, both /\textipa{g}/ and /\textipa{N}/) are not necessarily grouped together as a natural class because the two appear to be treated differently by different speakers.\footnote{The acoustic effect of a velar consonant on surrounding vowels is that F1 lowers, F2 raises, and F3 lowers, a phenomenon known as the \textit{velar pinch}. The degree to which this pinch affects a vowel is greater for /\textipa{N}/ than for /\textipa{g}/ because of the accompanying lowering of the velum \citep{baker_etal_2008}. Therefore, this tendency to raise, coupled with the auditory perception that low vowels are pereived as raised when nasalized (as discussed above), means pre-/\textipa{N}/ vowels have two forces acting upon them to encourage raising.} For example, \citet[46]{conn_2000_diss} finds that while Portlanders do have \ban-raising, for some speakers \bang is raised less than \ban and for others \bang is more fronted than \ban. Some studies separate all three front lax vowels, \kit, \dress, and \trap, into allophones that are followed by /\textipa{m}/ and /\textipa{n}/, and allophones followed by /\textipa{N}/. In other words, they analyze \bin, \ben, and \ban as distinct allophones from \bing, \beng, and \bang (which are all different from the elsewhere allophones \bit, \bet, and \bat). \citet[42]{cardoso_etal_2016_pads} were justified in this methodology because they found that \bing was raised more than \bin or \bit (see also \citealt{eckert_2004}), with women's realizations the highest, while \bin was not raised. They also find a significant effect for age on the height of \bang. For the purposes of this dissertation, I adopt this position and treat prevelar nasal environments distinct from pre(other)-nasal environments.

Summarizing the prenasal split, the role of nasal allophones and their relationship to the Elsewhere Shift is not as well-known as their oral counterparts. It is clear that the prenasal split applies to \trap and that the gap between the two allophones is widening in apparent time, but if the split applies to \dress and \kit, little evidence has been presented to support this. However, front lax vowels before /\textipa{N}/ do appear to be raising, at least in studies that have looked at them specifically, though the status of \beng is unclear due to the very low number of tokens containing that sequence. More work is needed to fully understand if and how prenasal tokens pattern together.

In Chapter \ref{ch:prenasal}, I shed light on the structural relationship of prenasal and pre-/\textipa{N}/ allophones of \trap, \dress, and \kit. The vowels' relationship to each other was tentative at best, suggesting that changes in prenasal environments in the West are perhaps driven by phonetics rather than by some larger phonological structure; changes are likely community-specific rather than being pan--North American.





\section{Social meaning in the Elsewhere Shift}
\label{sec:social_meaning_elsewhere_shift}

Finally, it is important to discuss the social meaning that is associated with the Elsewhere Shift. A large body of research has analyzed how listeners perceive shifted or unshifted variants, showing that people are sensitive to and assign social meaning to aspects of the Elsewhere Shift. Furthermore, speakers use these shifted variants and their associated meanings as a part of identity and persona construction.

Several studies have shown that the a retracted \bat vowel indicates a variety of social meanings to listeners. First, there is the negative association of the ``Valley Girl``, which is perceived as  shallow, materialistic, and unintelligent \citep[47]{donofrio_2016_diss}. Valley Girls, together with the male counterparts, ``Surfer Dudes'', were stereotyped in California-based songs, movies, and comedy in the 1980s and 1990s. Some of these stereotypes exist today, and a retracted \bat continues to index some of these attributes today.

However, an unrelated and contrastive social meaning associated with retracted \bat is that of professionalism and education. \citet[47]{donofrio_2016_diss} shows that speakers and listeners associate a backed or lowered \bat with formality, upper class, education, correctness. Overall they evaluated a more shifted \bat as evoking a business professional persona. In an in-depth analysis of a single woman's vowels, \citet{vanhofwegen_2017_diss} shows that speakers use stylized variants of vowels that are more peripheral, longer in duration, more likely to be creaky; specifically for \bat, the speaker used these stylized variants when ``taking a stance of knowledge during these interactions---she knows something her classmates do not'' \citeyearpar[149]{vanhofwegen_2017_diss}. \citet{podesva_etal_2012} analyzes the speech of Condoleeza Rice and show that she uses linguistic features that are associated with formality, being highly educated, and standardness, including released word-final voiceless obstruents and a backed \bat. \citet{donofrio_2018} points out that \bat may index these particular meanings because of how Americans typically view British English, and in particular, the \trap-\bath split. This lexically conditioned split is a part of some varieties of British English and the backer vowel is often perceived as more intelligent and correct. Indeed, \citet{boberg_1999} shows that American English listeners perceive /\textipa{a:}/ as more correct, educated, and sophisticated than /\textipa{\ae}/ in foreign words like \textit{llama}, \textit{pasta}, and \textit{drama}. These various factors combine to create an overall sense that the retracted \bat conveys a level of sophistication that the conservative front \bat does not.

% TODO: Dr. Renwick's comment on what is meant by wiggle room. I want to say it's because there aren't nearby vowels and that it's quite different acoustically from the others but I don't know if that's objectively true, and it sort of splits up the Pratt et al's description.
In addition to these primary meanings of ``Valley Girl'' and business professional persona, the indexical field for retracted variants of \bat include various additional meanings as well. \citet{geenberg_2014_diss} finds that speakers who had spent more time outside of their rural community in California used backer variants of \bat than those who did not leave the county. However, nearby in Redding, \bat was one of the few linguistic features that was \textit{not} associated with orientation towards the town verses the country \citep{podesva_etal_2015}. Among Chicano English speakers in Culver City, California, retracted \bat was used more by non-gang members than gang-members (and this distinction was more important than social class or language background), suggesting that these non-gang members are conforming more with the majority community as a part of their linguistic expression \citep{fought_2003}. \citet[150]{vanhofwegen_2017_diss} provides several examples of how a lowered \bat is used when a speaker expresses ``righteous indignation'' and calls for additional study on such extreme tokens to get a more complete picture of what these tokens mean. I believe \citeauthor{pratt_etal_2018}'s \citeyearpar{pratt_etal_2018} description of \bat describes it perfectly: there is a great deal of ``wiggle room,'' and speakers have been shown to exploit those different variants to serve a variety of multifaceted purposes.

% BAT outside of CA
For non-Californians, it appears that while a retracted variant of \bat carries less social meaning than it does in California, it is often associated with California itself. For example, in Oregon, \citet{adcock_becker_2016} find that listeners link a backed variant of \bat with California personae. And based on the perceptions of listeners from the Bay Area, Portland, and Seattle, \citet{becker_swan_2019} find the backed \bat was perceived as young and frivolous, which is possibly related to the Valley Girl stereotype that came out of California. Given the Californian stereotypes that are perpetuated with the Elsewhere Shift, these associations come as no surprise. In fact, based on the work of \citet{labov_1963}, \citet{eckert_2000}, and \citet{zhang_2005}, \citet[462]{eckert_2008_indexicalFields} shows that ``variables that historically come to distinguish geographic dialects can take on interactional meanings based in local ideology\ldots Local identity is never an association with a generic locale but with a particular construction of that locale as distinct from some other.'' In other words, we would not expect the full indexical field of retracted \bat to be the same across all areas of the West. Specifically, the ``business professional'' persona that is documented in California does not appear to transfer to other areas. However, its associations with California do.\footnote{That the shift occurs in California or indexes California may be reason enough to continue calling it the \textit{California Vowel Shift}. This association may be lost (or perhaps never existed) in areas far from the Pacific Coast though, so I will continue using the term \textit{the Elsewhere Shift}.}

Like any other variable, \bat has an indexical field that includes a variety of meanings, some of which are contradictory. Specifically, \citet{becker_swan_2019} also found that when listeners tried to guess where the speakers were from, retracted \bat was most correlated with being not from the West Coast, not from California, and possibly from Canada. In other words, the California-ness that some listeners assign to that variable is not universal. Instead, \citeauthor{becker_swan_2019} argue that, to these listeners, \bat retraction may just be a generic, supra-local, and unspecified feature. Additional work is needed on listener perception of the Elsewhere Shift to fully understand these social meanings.

% BAN-raising
In addition to \bat retraction, \ban-raising has been found to vary sociolinguistically in California. \citet{eckert_2008} focused on the nasal split in two schools separated by only a ten minute drive. In Fields Elementary, \ban is raised and in Steps Elementary, \ban is not. The majority of students at Fields are middle-class Anglos while students at Steps come from a poorer, ethnically diverse population where Chicano English has the most linguistic capital. But a lack of raising is more than just an ethnicity difference because it is used by children of all ethnicities, particularly to index the ``coolness that emerge[d] within the ethnic group'' \citeyearpar[41]{eckert_2008}. In a more rural part of California, Redding, \citet{podesva_etal_2015} found that a higher \ban vowel was used by younger, country-oriented males (as opposed to ``townies''). Given that the prenasal split is strongest among urban areas, it is somewhat surprising to find that this group lead the change. But \citeauthor{podesva_etal_2012} argue that because \ban-raising has become a pan-regional pattern of American English, these speakers' use of this new national norm is a result of their opposition towards the big California cities, even though a more extreme form of \ban raising is a part of California English.


% BET and BIT
While the amount of social meaning associated with \trap (that is, both \bat and \ban) is extensive, \dress and \kit appear to be less socially salient. For example, in their study of Condoleeza Rice's speech, \citet[76]{podesva_etal_2012} find that she did not shift these two vowels to indicate formality as she did with \trap. Similarly, some studies find that while other aspects of the Elsewhere Shift are advanced when constructing a particular identity, \bet and \bit do not change \citep{fought_2003, podesva_2011}. Nevertheless, in stereotyped parodic performances of Californians, \textit{Saturday Night Live} comedian Kristen Wiig uses significantly backed variants of all three vowels, though this is likely the result of her open-jawed setting that she uses extensively while portraying that character rather than social meaning of the vowels themselves, because in those same skits co-performer Fred Armisen does not shift his vowels in the same direction but does employ a distinctive jaw setting and lip protrusion \citep{pratt_donofrio_2017}.



% Moving to the low vowel
Moving to the low back merger, \citet{eckert_labov_2017} find that people do not generally associate social meaning to abstract phonological processes, like a vowel merger. They do, however, find that ``social meaning attaches to the individual shifts that bring about the merger'' \citeyearpar[484]{eckert_labov_2017}. This holds true with the Third Wave studies on the low back vowels. For example, \citep{pratt_2018} finds that tech students at a California high school used a higher \lot vowel than the other students did, indexing the ``toughness'' that is a part of being in that social group. On the other hand, \citet{hall_lew_2013} suggests that social meaning is associated with \thought, which is the reason for it being more fronted than \lot (resulting in a flip-flop in the vowel space). Exactly what that meaning is is difficult to pin down:
\begin{quote}
    ``[I]t may be that more advanced [{\thought}] tokens are a component of `Asian' styles, or just new local persona more generally. Perhaps an advanced [{\thought}] vowel was just one small part of the stylistic package that indexed `five-foot-tall Asian girl[s] who could breakdance'.'' \citet[381]{hall_lew_2013}
\end{quote}
In that study, what the three women that had the most fronted \thought shared was ``a lifetime of active negotiation between conflicting local authenticities'' \citeyearpar[386]{hall_lew_2013}. Similarly nuanced and complex meanings may be associated with the shifting low back vowels in other Western communities and additional work is needed to fully understand this variable.

% Overall vowel space
Finally, rather than dissecting the Elsewhere Shift into its constituent parts, several studies have shown that multiple linguistic variables shift in concert to index specific social meanings. For example, \citet{pratt_2018} shows that the ``toughness'' in the tech students mentioned above was also indexed with a more velarized /\textipa{l}/ and that the combination of the two is what conveys the social meaning. In an in-depth study of one gay California man's speech, \citet{podesva_2011} shows that a more advanced \bat-retraction, \ban-raising, and back vowel fronting are all used to in some situations to help the speaker construct a gay ``partier'' persona. He argues that it is the ``ways in which variables are combined and packaged'' (\citealt[41]{podesva_2011}, see also \citealt{campbellkibler_2011}) that index social meaning.


%Finally, among Californians, \citet[68]{villarreal_2016_pads} finds that that no part of the Elsewhere Shift was significantly correlated correctness, pleasantness, self-similarity, and place.

To summarize, many aspects of the Elsewhere Shift index a wide variety of social meanings. In California, retracted \bat is associated simultaneously with the ``Valley Girl'' and the ``business professional'' personae, while elsewhere it is often associated with California itself. \ban-raising, and both \lot and \thought have more varied meanings, and \bet and \bit have relatively little social meaning. Additional research will help us fully explore the ``wiggle room'' and socioindexical variation ascribed to these sounds.

\section{Conclusion}

% TODO: Renwick: Q. WE NEED A SCHEMATIC PLOT HERE... (Please draw it on the board) --> actually this should just be the Elsewhere Shift schematic from your presentation slide 2
For the purposes of this study, the Elsewhere Shift will be defined as the lowering and/or retraction of the front lax vowels, the raising of \ban, and the merger or near-merger of the low back vowels (possibly accompanied with raising). Numerous studies have documented the presence of all or parts of the Elsewhere Shift in many regions of North America. However, the precise nature of the shift (both the trajectory of change and its relative timing) exhibits a large amount of variation, which has led to different conclusions about the structural relationship between the shifting vowels.

%TODO Q. It is not clear to me WHICH VOWELS YOU WANT TO STUDY. There seems to be no set of "research questions" laid out, and there is no chapter or section heading called "research questions" or "goals" or anything similar -- instead, it's as if you are just reacting to others' work, when in fact you have done an enormous amount of thinking and analysis. Make it clear what your contribution sets out to be, and give us signposts to it.
%TODO I wonder if a solution could come from my other comment about whether people identify more with Seattle or with Portland. For instance, you can explicitly test (a) whether Cowlitz Countians participate in the Elsewhere Shift, and (b) whether that has participation has increased over generational time. If (a) is true, this is evidence that people lean more towards Portland than Seattle, which perhaps is surprising given that CC is in Washington State (but Portland is closer geographically); if (b) is true, this is evidence in favor of your argument that social/economic change in CC has led to, or is related to, linguistic change over time.
